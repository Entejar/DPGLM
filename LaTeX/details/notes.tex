\documentclass{article}[12pt]
\usepackage[utf8]{inputenc}
\usepackage{amssymb}
\usepackage{mathtools}
\usepackage{multirow}
\usepackage{textcomp}
\usepackage[a4paper, margin=1 in]{geometry}
\usepackage{graphicx}
\usepackage{hyperref}
\usepackage{bm}
\usepackage{xcolor}
\definecolor{LightGray}{gray}{0.9}
\usepackage{float}
\usepackage{mathrsfs}
\usepackage{natbib}
\usepackage{longtable}
\bibliographystyle{plainnat}
\usepackage{theorem}
\newcounter{theorem}
\renewcommand\thetheorem{\arbic{theorem}}
\newtheorem{theorem}{Theorem}%[theorem]
%\newtheorem{algorithm}{Algorithm}%[theorem]
\newtheorem{corollary}{Corollary}%[section]%[theorem]
\newtheorem{fact}{Fact}%[section]
\newtheorem{property}{Property}%[section]
\newtheorem{result}{Result}%[section]
\newtheorem{lemma}{Lemma}%[section]
%\newtheorem{proof}{Proof}
%%%
\newtheorem{assumption}{Assumption}

%%%%%%%%%%%%%%%%%%%%%%%%%%%%%%%%%%%%%%%%%%%%%%%%%%%%
%\newcommand{\estimates}{\overset{\scriptscriptstyle\wedge}{=}}
\newcommand{\limd}{\lim_{\delta \rightarrow 0}}
\newcommand{\limn}{\lim_{n \rightarrow \infty}}
\newcommand{\limns}{\limsup_{n \rightarrow \infty}}
\newcommand{\limni}{\liminf_{n \rightarrow \infty}}
\newcommand{\ddelt}{\frac{d}{dt}}
\newcommand{\intrp}{\int_0^\infty}
\newcommand{\intr}{\int_{-\infty}^\infty}
\newcommand{\PP}{\mathbb{P}}
\newcommand{\RR}{\mathbb{R}}
\newcommand{\EE}{\mathbb{E}}
\newcommand{\II}{\mathbb{I}}
\newcommand{\LL}{\mathcal{L}}
\newcommand{\sumni}{\sum_{i = 1}^n}
\newcommand{\sumNi}{\sum_{i = 1}^N}
\newcommand{\summr}{\sum_{r = 1}^m}
\newcommand{\summj}{\sum_{j = 1}^m}
\newcommand{\sumnj}{\sum_{j = 1}^n}
\newcommand{\dto}{\stackrel{D}{\rightarrow}}
\newcommand{\Pto}{\stackrel{P}{\rightarrow}}
\newcommand{\asto}{\stackrel{a.s.}{\rightarrow}}
\newcommand{\dd}{\stackrel{\text{d}}{=}}
\newcommand{\del}{\partial}
\newcommand{\eps}{\varepsilon}
\renewcommand{\epsilon}{\varepsilon}
\newcommand{\SA}{\mathscr{A}}
\newcommand{\CA}{\mathcal{A}}
\newcommand{\CF}{\mathcal{F}}
\newcommand{\iid}{\stackrel{\mathrm{iid}}{\sim}}
\newcommand{\ind}{\stackrel{\mathrm{ind}}{\sim}}
\definecolor{darkgreen}{rgb}{0,0.5,0}


%%%% citations style
\newcommand{\customcite}[1]{\citeauthor{#1}, \href{cite.#1}{\textcolor{blue}{\citeyear{#1}}}}

\newcommand{\customcitetwo}[1]{\citeauthor{#1} (\href{cite.#1}{\textcolor{blue}{\citeyear{#1}}})}

\newcommand{\customcitet}[2]{(\href{cite.#1}{\citeauthor{#1}}, \href{cite.#1}{\textcolor{blue} {\citeyear{#1}}}; \href{cite.#2}{\citeauthor{#2}}, \href{cite.#2}{\textcolor{blue} {\citeyear{#2}}})}

\newcommand{\todo}[1]{\textcolor{red}{[To Do: \textcolor{teal}{#1}]}}
\newcommand{\customref}[1]{\textcolor{blue}{\ref{#1}}}

\newcommand{\Ga}{\mbox{Ga}}
\renewcommand{\th}{\theta}
\newcommand{\cb}{\bm{c}}
\renewcommand{\sb}{\bm{s}}
\newcommand{\vb}{\bm{v}}
\newcommand{\wb}{\bm{w}}
\newcommand{\bx}{\mbox{\boldmath $x$}}
\newcommand{\beps}{\mbox{\boldmath $\epsilon$}}
\newcommand{\estimates}{\mathrel{\widehat{=}}}
\newcommand{\tmu}{\widetilde{\mu}}
\newcommand{\scf}{\mathcal{F}}
\newcommand{\sx}{\mathcal{X}}
\newcommand{\sy}{\mathcal{Y}}
\newcommand{\bth}{\mbox{\boldmath $\theta$}}

\renewcommand{\baselinestretch}{1.8}

\title{DP--SPGLM Notes}
\author{Entejar}
\date{}

\begin{document}

\maketitle

\section{Introduction}
(Model): $p(y_i \mid x_i = x) \sim p_x \propto \exp(\th_x y_i)\tmu(y_i)$ with $y_i \in \mathcal{Y}$ and $x \in \mathcal{X}$. 
\vspace{1mm}
\newline (Prior): $\tmu \sim$ CRM (completely random measure) e.g., gamma CRM (normalized version of this is DP). CRM is characterized by Levy intensity $\nu(s, y) = \rho(s \mid y) H(y), s \in \mathcal{R}^+, y \in \mathcal{Y}$ and is almost surely discrete in nature:
$$
\widetilde{\mu}(\cdot)=\sum_{h=1}^{\infty} \omega_h \delta_{m_h}(\cdot)
$$
This implies a prior on $\scf = \{p_x: x \in \sx\}$. With gamma CRM on $\tmu$, this becomes the Dependent Dirichlet Process (DDP) prior.
\vspace{1mm}

\noindent Note that $\exists$ lot of literature on DDP priors, with typical ``common weight DDP'':
\begin{equation}
p_x(\cdot) = \sum_{h=1}^\infty \omega_{h} \delta_{m_h(x)}(\cdot) \label{varyingweightsDDP}
\end{equation}
The selling points of our model and prior are:

-- dependence is on weights $\omega_h(x)$ with common atoms, which is not very common in literature.

-- inhomogeneous CRM on $p_x$ [this is cute, but you might not care].

\section{Posterior simulation}
(James et al): likelihood, $y_i \sim p \propto \tmu$ with prior on $\tmu \sim$ CRM. Then, they did posterior simulation by introducing a latent variable $u \in \mathcal{R}^+$. Posterior $\tmu \mid u, \mathbf{y} \sim$ again a CRM with updated parameters. 
\vspace{2mm}
\newline (Our setup): likelihood $p_x(y_i) \propto \exp(\th_x y_i)\tmu(y_i)$ and CRM prior on $\tmu$. We think we can show and do posterior simulation by introducing a $n$ latent variables $\mathbf{u} = (u_1, \dots, u_n) \in {\mathcal{R}^+}^n$. Then, posterior $\tmu \mid \mathbf{u}, \bth, \mathbf{y} \sim$ CRM again.
\vspace{2mm}
\newline (Notes):

-- The above is true only for fixed $\th_i$.

-- Maybe we can add one more layer to the hierarchical structure: $\theta_i \sim N(\phi_i, c)$ with very small $c$ and $\phi_i = \phi(\beta, \tmu, x_i)$. In this case, we can do Gibbs update: $\beta \to \bth \to \mathbf{u} \to \tmu \to \beta \to \dots$

-- Easy to generalise to DPM.

\end{document}

