\documentclass[10pt, oneside]{article}   	% use "amsart" instead of "article" for AMSLaTeX format
\usepackage{geometry}                		% See geometry.pdf to learn the layout options. There are lots.
\geometry{letterpaper}                   		% ... or a4paper or a5paper or ... 
%\geometry{landscape}                		% Activate for rotated page geometry
%\usepackage[parfill]{parskip}    		% Activate to begin paragraphs with an empty line rather than an indent
\usepackage{graphicx}				% Use pdf, png, jpg, or eps§ with pdflatex; use eps in DVI mode
								% TeX will automatically convert eps --> pdf in pdflatex		
\usepackage{amssymb}
\usepackage{mathtools}
\usepackage{amsfonts}
\usepackage{xcolor}
\usepackage{ulem}
\usepackage{cancel}
\usepackage{amsthm} % for theorem environments
%SetFonts

%SetFonts
\newcommand{\Q}[1]{\textcolor{brown}{[Q: \textcolor{teal}{#1}]}}
\newcommand{\citea}[1]{\citeauthor{#1}, \href{cite.#1}{\textcolor{blue}{\citeyear{#1}}}}
\newcommand{\citeb}[1]{\citeauthor{#1} (\href{cite.#1}{\textcolor{blue}{\citeyear{#1}}})}
\newcommand{\citec}[2]{(\href{cite.#1}{\citeauthor{#1}}, \href{cite.#1}{\textcolor{blue} {\citeyear{#1}}}; \href{cite.#2}{\citeauthor{#2}}, \href{cite.#2}{\textcolor{blue} {\citeyear{#2}}})}

\newcommand{\todo}[1]{\textcolor{red}{[To Do: \textcolor{teal}{#1}]}}
\newcommand{\refa}[1]{\textcolor{blue}{\ref{#1}}}



\newcommand{\Ga}{\mbox{Ga}}
\renewcommand{\th}{\theta}
\newcommand{\cb}{\bm{c}}
\renewcommand{\sb}{\bm{s}}
\newcommand{\vb}{\bm{v}}
\newcommand{\wb}{\bm{w}}
\newcommand{\bx}{\mbox{\boldmath $x$}}
\newcommand{\beps}{\mbox{\boldmath $\epsilon$}}
\newcommand{\estimates}{\mathrel{\widehat{=}}}
\newcommand{\thh}{\widetilde{H}}
\newcommand{\scf}{\mathcal{F}}
\newcommand{\sx}{\mathcal{X}}
\newcommand{\sy}{\mathcal{Y}}
\newcommand{\bth}{\mbox{\boldmath $\theta$}}
\newcommand{\E}{\mathbb{E}}
\newcommand{\R}{\mathbb{R}}
\renewcommand{\scf}{\mathcal{F}}
\renewcommand{\sx}{\mathcal{X}}
\renewcommand{\sy}{\mathcal{Y}}
\newcommand{\syy}{\mathbb{\sy}}
\newcommand{\sbb}{\mathcal{B}}
\newcommand{\sd}{\mathcal{D}}
\newcommand{\eps}{\epsilon}
\newcommand{\pr}{\mbox{Pr}}
\newcommand{\ub}{\mathbf{u}}
\newcommand{\tn}{{\widetilde n}}
\newcommand{\tmu}{\widetilde{\mu}}

\newcommand{\red}{\color{red}}
\newcommand{\blu}{\color{blue}}
\newcommand{\bla}{\color{black}}
\newcommand{\bch}{\blau\it}
\newcommand{\ech}{\schwarz\rm}
\newcommand{\cut}{{\blau \bf [CUT] }}
\newcommand{\mynote}[1]{\footnote{#1}}


% Theorem styles
\newtheorem{result}{Result}[section] % Numbered within each section
\newtheorem{theorem}{Theorem}[section] % Numbered within each section
\newtheorem{lemma}[theorem]{Lemma} % Numbered with theorems
\newtheorem{corollary}[theorem]{Corollary} % Numbered with theorems

% Proof environment
\renewenvironment{proof}{\noindent \textit{Proof.}}

\title{\bf DP-GLM and Posterior Sampling Steps}
%\author{}
\date{}							% Activate to display a given date or no date

\begin{document}
\maketitle

\section{DP-GLM}
\begin{align}
 y_i \mid  z_i, x_i & \sim K(y_i \mid z_i, x_i) = K(y_i \mid z_i), \quad y_i, z_i \in \sy \\
z_i \mid x_i = x, \widetilde \theta_{x}, \tmu & \sim p_x(z_i) \propto
                                       \exp(\widetilde \th_{x} z_i) \widetilde
                                       \mu(z_i) \\ 
\widetilde \theta_x \mid \theta_x & \sim p(\widetilde \theta_x \mid \theta_x), \text{ with } b^\prime(\theta_x) = \int_\sy z \frac{\exp( \th_{x} z) \widetilde
                                       \mu(z)}{\int_\sy \exp(\th_{x} u ) \widetilde
                                       \mu(u) du} dz = g^{-1}(x^\prime \beta)  \\ 
\tmu & \sim \text{ gamma CRM}(\nu), \text{ with } \nu(ds, dz) = \frac{e^{-s}}{s} ds \cdot \alpha G_0(dz) \\ 
\beta & \sim \text{ MVN}(\mu_\beta, \Sigma_\beta). 
\label{dpglm}
\end{align}
\subsection{Notes}
\begin{itemize}
    \item Eq. (4) implies $\frac{\widetilde \mu}{\tmu(\sy)}  \sim DP(\alpha, G_0).$
    \item Symmetric $K(\cdot \mid z)$ in $z$ respects the GLM mean regression. $\E(y_i \mid
x_i) = \E_{z_i, \widetilde \th_i \mid x_i}  \E(y_i \mid x_i, z_i, \widetilde \th_i) = \E_{z_i, \widetilde \th_i \mid x_i} (z_i) = \E_{\widetilde \th_i \mid x_i} \E_{z_i \mid x_i, \widetilde \th_i} (z_i) = \E_{\widetilde \th_i \mid x_i} (b^\prime (\widetilde \th_i)) = b^\prime (\th_i) = g^{-1}(x^\prime_i \beta)$. \red The second last equality holds only if $b^\prime$ is a linear function. \bla
\item $p(\widetilde \theta_x \mid \theta_x) = N(\theta_x, \sigma^2_\th)$ makes life easy! Closed form complete conditional for $\widetilde \theta_x$. We choose $\sigma_\th$ to be small positive fraction. This ensures $\E(y_i \mid
x_i) = g^{-1}(x^\prime_i \beta) + R_i \approx g^{-1}(x^\prime_i \beta)$, where $R_i \approx 0$. 
\item Choices to be made for $\alpha, G_0, K, \mu_\beta, \Sigma_\beta, g$.
\end{itemize}



\subsection{Modeling fractional data}
Here $\sy = [0, 1]$.
\begin{itemize}
    \item $K(\cdot \mid z_i)  = \text{ Uniform}(z_i - c_0, z_i + c_0)$. 
    \item $G_0 = \text{ Uniform}(0, 1)$.
    \item $g(\mu) = \ln(\frac{\mu}{1 - \mu})$ [logit link]
\end{itemize}
  

\section{Posterior sampling steps}
\subsection{Notes}
\begin{itemize}
    \item Model parameters: $(\beta, \widetilde \mu, \widetilde \theta_i)$. \red Is $\widetilde \theta_i$ a model parameter or latent variable? \bla
    \item Latent variables: $\{z_i, u_i\}_{i=1}^n$.  
    \item Derived parameters: $\th_i = {b^\prime}^{-1} (g^{-1}(\eta_x)$), where $\eta_x = x^T\beta$ or $s(x)^T\beta$. We can go fully nonparametric here; bad thing is, we lose the $\beta$ interpretation.
    \item  $M$ is the CRM finite truncation point. We use $M = 20$.
\end{itemize}
\subsection{Steps}
\begin{itemize}
    \item[(A):] $[\widetilde \beta \mid -] \sim N_p(\mu^\star_\beta, \Sigma^\star_\beta) 1_A\left(\widetilde{\beta}\right)$ is the proposal for updating $\beta$, where $\mu^\star_\beta = \beta^{(t)}, \Sigma^\star_\beta = \rho  \mathcal{I}^{-1}\left(\widehat \beta_{mle} ; \widehat {\widetilde \mu}_{mle}\right)$, where $A = \left\{\beta \in \R^p :  g^{-1}\left(x_i^T \beta \right) \in \sy \right \}$, $\rho \in (0,1]$ is a tuning parameter, and $ \mathcal{I}^{-1}\left(\widehat \beta_{mle} ; \widehat {\widetilde \mu}_{mle}\right)$ is the inverse Fisher information matrix associated with $\beta$ at $(\beta, \tmu) = \left(\widehat \beta_{mle} ; \widehat {\widetilde \mu}_{mle}\right)$, where
    \[\mathcal{I}\left(\beta ; {\widetilde \mu}\right) = \sum_{i=1}^n  \frac{x_i x_i^T}{\left(g^{\prime}\left(\mu_i\right)\right)^2 b^{\prime \prime}\left(\theta_i\right)} \ ,\]
    $\mu_i = \E(y_i \mid x_i) = g^{-1}(x^T_i \beta)$ and $\th_i = \th(x_i; \beta, \tmu)$ is the derived parameter. Hence the proposal distribution implicitly depends on $\tmu$ through dispersion matrix. We correct the MH step with an acceptance probability,
    $$
    \delta(\widetilde \beta \mid \beta) = \frac{L(\widetilde \beta) \pi(\widetilde \beta)}{L(\beta) \pi(\beta)} \cdot \frac{q(\beta \mid \widetilde \beta)}{q(\widetilde \beta \mid \beta)},
    $$
    where 
    \begin{align*}
        L(\beta, \tmu) & = \prod_{i=1}^n p(y_i \mid x_i; \beta, \tmu) \\
        & = \prod_{i=1}^n \int_{\widetilde \th_i} \int_{z_i} p(y_i \mid z_i, \widetilde \th_i, x_i; \beta, \tmu) \cdot p(z_i, \widetilde \th_i \mid x_i; \beta, \tmu) dz_i  d{\widetilde \th_i}\\
        & = \prod_{i=1}^n \int_{\widetilde \th_i} \int_{z_i} p(y_i \mid z_i) \cdot p(z_i \mid \widetilde \th_i, x_i; \beta, \tmu) \cdot p(\widetilde \th_i \mid x_i; \beta, \tmu) dz_i  d{\widetilde \th_i}\\
        & = \prod_{i=1}^n \int_{\widetilde \th_i} \int_{z_i} K(y_i \mid z_i) \cdot p(z_i \mid \widetilde \th_i; \tmu) \cdot p(\widetilde \th_i \mid \th_i) dz_i  d{\widetilde \th_i},
    \end{align*}
    with $K(y_i \mid z_i) = \frac{1}{2c_0} 1(z_i - c_0 \leq y_i \leq z_i + c_0)$, \ $p(z_i = z \mid \widetilde \th_i, \tmu) = \frac{\exp(\widetilde \th_i z)\widetilde \mu(z)}{\int_\sy \exp(\widetilde \th_i u) \widetilde \mu(u) du}$, \ $p(\widetilde \th_i \mid \th_i) = \text{ Normal}(\widetilde \th_i \mid \th_i, \sigma^2_\th)$, $\th_i = {b^\prime}^{-1} (g^{-1}(x^T_i\beta))$. Note that $L(\beta \mid \tmu)$ is the conditional likelihood contribution of $\beta$ given $\tmu$. So, given $\widetilde \mu(\cdot) = \sum_{m = 1}^M J_m \delta_{s_m}(\cdot)$, \ $p(z_i = s_m \mid \widetilde \th_i, \tmu) = \frac{\exp(\widetilde \th_i s_m) J_m}{\sum_{m'=1}^M \exp(\widetilde \th_i s_{m'}) J_{m'}}$. 
    \red Do we need to use marginal (instead of conditional) likelihood contribution of $\beta$? \bla $\pi$ and $q$ are the prior and proposal density for $\beta$.
    
    \item[(B):] $[\widetilde \theta_{x_i} \mid -] ~ \text{ Normal}(\theta_{x_i} + z_i \sigma^2_\th, \sigma^2_\th)$, where $\th_{x_i} = \th(x_i; \beta, \tmu)$ is the derived parameter.
\item[(C):] $[\ub \mid -]:$
\begin{eqnarray*}
Pr\left(\ub \mid  \mathbf{z}^{(t)}, \mathbf{\theta}^{(t)} \right) \propto \exp \left \{ - \int_{\sy} \ln \left[1+ \sum_{i=1}^n u_i \exp(\th^{(t)}_i  z) \right] G_n(dz) \right\}, 
\end{eqnarray*}
where $G_n(\cdot) = \alpha G_0(\cdot) +  \sum_{j=1}^n \delta_{z^{(t)}_j} (\cdot)$. The second term can also be written as $\sum_{\ell = 1}^{\widetilde n} n_\ell \delta_{z^\star_\ell}(\cdot)$, where $\{z^\star_\ell\}_{\ell=1}^{\widetilde n}$ are the unique values, among $\{z^{(t)}_j, j=1, \dots, n\}$, with $\{n_\ell\}_{\ell=1}^{\widetilde n}$ ties. \vspace{2mm}

\textit{Proposal for} $\ub$ [Idea from Igor's Stat Sci 2013 paper]: We generate proposal using random walk, $u_j \sim Gamma\left(\delta, \delta\big/u^{(t)}_j\right), j = 1, \dots, n$ and followed up with an MH acceptance-rejection step. The hyperparameter $\delta (\geq 1)$ controls the acceptance rate of the M-H step being higher for larger values.

\item[(D):] $[\widetilde \mu \mid \mathbf{u}^{(t+1)}, \mathbf{z}^{(t)}, \mathbf{\theta}^{(t)}]:$
\begin{itemize}
 \item[(a)]  ${\widetilde \mu}^{\left({\ub}^{(t+1)}\right)} \sim \text{ CRM}\left(\nu^{(\ub)}\right) \text{ with } \nu^{(\ub)} (ds, dz) = e^{- \left\{\sum_{i=1}^n u^{(t+1)}_i \exp(\th^{(t+1)}_i z) \right\} s} \rho(ds \mid z) G_0(dz) = e^{- \Psi(z) s} \rho(ds \mid z) G_0(dz) = \alpha \frac{e^{- \left(\Psi(z) +1\right) s}}{s} ds G_0(dz)$.  ${\widetilde \mu}^{\left({\ub}^{(t+1)}\right)}(\cdot) = \sum_{\ell =1}^M J^{(t+1)}_\ell \delta_{z^{(t+1)}_\ell}(\cdot)$ (random locations $z_\ell$ and jumps $J_\ell$ are generated using Ferguson and Klass (1972) algorithm; $M$ is used for finite approximation of the CRM). Omitting superscript $u$ for better readability.
 $$
 N(v) = \nu([v, \infty], \sy) = \int_v^\infty  \int_{\sy} \nu(ds, dz) = \alpha \int_v^\infty  \int_{\sy} \frac{e^{- \left(\Psi(z) +1\right) s}}{s} ds G_0(dz) 
 $$
 
 \item[(b)] $\Pr\left(J^{\left({\ub}^{(t+1)}, z^{(t)}_j\right)}_j\right) \propto s e^{-s \left\{\sum_{i=1}^n u^{(t+1)}_i \exp(\th^{(t+1)}_i z^{(t)}_j)\right\}} \rho(ds \mid z^{(t)}_j) =  e^{- \left(\Psi\left(z^{(t)}_j\right) +1\right) s} \sim EXP$ $\left(mean = 1\bigg/\left(\Psi\left(z^{(t)}_j\right) +1\right)\right)$. Let's assume we have $r$ unique fixed locations ($z^\star_1, \dots, z^\star_r$). Then for jumps, we generate from $\Pr(J^\star_\ell \mid -)\propto s^{n_\ell - 1}  e^{- \left(\Psi\left({z^\star_\ell}^{(t)}\right) +1\right) s} \equiv Gamma\left(n_\ell,  \Psi\left({z^\star_\ell}^{(t)}\right) +1\right)$ $, \ell = 1, \dots, r$.
\end{itemize}
\item[(E):] $[\mathbf{z} \mid {\widetilde \mu}^{(t+1)}, \mathbf{y}]:$ 
\begin{eqnarray*}
\Pr[z_i \mid {\widetilde \mu}^{(t+1)}, \mathbf{y}] \propto \sum_{\ell} K(y_i \mid z_i) \exp(\theta_i z_i) {\widetilde{J}}^{(t+1)}_\ell \delta_{{\widetilde z}^{(t+1)}_\ell}(z_i)\ ,
\end{eqnarray*} 
where $\widetilde J := \{{\widetilde{J}}^{(t+1)}_\ell \}_{\ell \geq 1} = \{{J^\star_1}^{(t+1)}, \dots,  {J^\star_r}^{(t+1)}, {J_1}^{(t+1)}, \dots, \}$ [first $r$ jumps are from 2(b) and rest from 2(a)]. Similarly, $\widetilde Z := \{{\widetilde z}^{(t+1)}_\ell\}_{\ell \geq 1} = \{{z^\star_1}^{(t+1)}, \dots,  {z^\star_r}^{(t+1)}, {z_1}^{(t+1)}, \dots, \}$. Hence, $z_i \sim a \ categorical \ distribution$. 
\end{itemize}

\end{document}